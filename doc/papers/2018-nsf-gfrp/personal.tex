\documentclass[12pt]{article}

\usepackage{hyperref}
\usepackage[margin=1in]{geometry}
\usepackage[small, compact]{titlesec}

\usepackage[
  sorting=none
]{biblatex}
\bibliography{references}
\ExecuteBibliographyOptions{maxbibnames = 5}
\usepackage{fixme}
\fxsetup{
  author = TODO,
  status = draft,
  layout = inline,
  theme = color
}

\usepackage{fancyhdr}
\pagestyle{fancy}
\fancyhf{}
\lhead{Personal Statement, Relevant Background, and Future Goals}
\rhead{Max Willsey}

% 3 page limit
\begin{document}

Programming languages connect people with an idea to the hardware that makes it possible.
Powerful languages help users reason about their code, letting them build faster and safer systems with less effort.
In complex domains, the right abstractions make the difference between programming being productive or nearly impossible.

Yet many areas remain out of reach for programming languages, leaving users without the safety and productivity benefits these tools can provide.
In some cases, the domain is too complicated for the currently proposed abstractions.
In others, users struggle with a complex programming model or the idea of programming at all.

I apply the right programming language techniques to problems across domains.
By collaborating with stakeholders and experts in various areas, I work to balance the power of programming systems with their usability.
The research is rich with intellectual challenges, but it's fundamentally about empowering people to tackle more problems more efficiently.

With an NSF Fellowship, I will continue working across domains to give users the powerful tools they need.
The collaborative nature of the work plus my personal interests in mentorship and teaching will lead to my long term goals of a professorship.
Leading a research team will give me the resources and opportunity to explore new domains where programming languages can help the most.

\section*{Research Experience -- Intellectual Merit}

My research clearly focuses on using programming languages to address difficulties and inefficiencies across domains.
My past and current projects demonstrate that I can collaborate with domain experts and create practical, usable systems.
These successes depended on choosing the right tool for the job and solving the problem in a way that's accessible to those in the field.
Awards and fellowships further confirm the intellectual merit of the work.

% The intellectual merit of the work on programming digital microfluidics is detailed in my research proposal.
% Beyond that, the proposed work will help develop the field of hybrid molecular-electronic systems.

% While the proposed research is in a different application domain than my past work, the core methods are similar.
% To succeed in the proposed work, I'll need to collaborate with experts in another domain to build a real-world system that depends on programming language principles.
% My past work has hinged on those same factors, and it has been recognized through awards and fellowships.
% So I'm familiar with what it takes to succeed in this kind of work.


\paragraph*{Concurrent systems}
Since my first programming languages class at CMU, I have sought opportunities to use those principles to make complex programming easier.
Course work introduced me to the challenges of concurrent programming, like race conditions and deadlocks.
Session types can alleviate these issues by giving information about the structure of communication, but existing work had yet to focus on usability or performance.

I worked with Frank Pfenning to create a programming language that uses session types to make concurrent programming easier and more efficient.
We collaborated on the language design with the goal of making it accessible to introductory computer science students.
I also implemented a runtime system that uses the guarantees from the type system to improve performance with intelligent scheduling and memory management decisions.

In our publication \cite{cc0}, we demonstrated that our implementation outperformed other traditional message passing systems.
The project culminated in my senior honors thesis which won the ``Exemplary Thesis'' award. The language is still used in both undergraduate and graduate courses at CMU.
Moreover, the project solidified my interest in work that uses programming language principles to build tangible, useful artifacts.

\paragraph*{Accelerator design}
My undergraduate research experience informed my search for a PhD program; I looked for colleagues who were also driven to solve problems with useful tools.
At the University of Washington, I am glad to be surrounded by researchers who look to theory---programming language or otherwise---with the goal of real-world impact.

In my first year, I started a project that uses techniques from program synthesis to aid the design of domain-specific hardware.
Because of the full-stack nature of the project, I am collaborating with a PhD student specializing in accelerator design and professors from both hardware and software backgrounds.
This project is ongoing, and our proposal and preliminary results have been honored with a Qualcomm Innovation Fellowship award.

\paragraph*{Molecular systems}
The Molecular Information Systems Lab at UW offers an exciting domain for application of programming languages, so I joined at the end of my first year.
Building on the lab's work in DNA storage and computing, we created a vision of hybrid molecular-electronic systems.
These systems will pair the enormous storage density and parallelism of DNA with the flexibility of electronic computing.
To motivate this paradigm, we proposed a hybrid molecular-electronic system for massively parallel image search with a capacity that would far exceed purely electronic systems \cite{molecularelectronic}.

In my current and future research, I want to bring these systems closer to reality.
My research proposal addresses a critical gap in programming the interaction between the molecular and electronic portions of these systems.
The MISL group is the perfect setting for this work, because
I can collaborate with experts in related fields like electrical engineering and synthetic biology.
% Also, the need for this work goes far beyond enabling hybrid molecular-electronic systems.
% I want to explore applications ranging from cheaper, safer medical diagnostics to a more interactive approach to education in the lab.
% In short, the intellectual challenges, potential impact, and available resources and collaborations make this an ideal project for me.
With this excitement, I've already begun preliminary work that's been met with excitement from MISL scientists and other researchers in the field.

% While hybrid molecular-electronic systems are promising, many research questions must be solved before they become practical.
% For one, communication between the molecular and electronic domain depends on a microfluidic system that can efficiently manipulate fluids with various molecular properties.
% In my research statement, I propose a system for safe, high-level microfluidic programming that will bring these hybrid molecular-electronic systems closer to reality.

\section*{Personal Experience -- Broader Impact}

Aside from research, I have consistently involved myself in leadership and service throughout undergrad, and I continue to do so during my PhD.
I've had great teachers and mentors throughout my academic career, so I look to give that experience back to younger students.

These experiences have been personally rewarding, and my own mentors and teachers have recognized the broader impact of my actions.
At CMU, I was honored with a Senior Leadership Award, and I was chosen as an Andrew Carnegie Scholar.
Both awards consider leadership in and out of academics.

I'm also excited to connect these personal activities with my research.
My current and upcoming work afford the opportunity to work with undergraduate students and develop technology for the classroom.

\paragraph{Mentoring}
Since high school, I've fortunately been welcomed into each new stage in my career, so I've sought to provide this experience to others through mentoring and orientation efforts.
At CMU, I was a resident assistant for two years, responsible for mentoring and advising approximately 30 first-year undergrads.
Many of my residents went on to become resident assistants on their own, citing my mentorship as a key motivation.
In addition, faculty and staff selected me to be on numerous panels and committees with goals ranging from fundraising to addressing campus sexual harassment and violence.
My leadership awards further support my impact on the CMU community outside of research.

In graduate school, I recognized the challenge of transitioning from undergrad to the PhD program, so I've focused on improving this experience.
I organized and ran the orientation for new PhD students, and many of them have expressed their appreciation for my efforts.
I'm also a part of the department-wide mentoring program, where current PhD students help first-years navigate and get acclimated to the program.

Finally, I've began to connect my personal passion for mentoring with my research.
I am working with a senior undergrad and a high school student on the digital microfluidics project detailed in my research proposal.
Leading them through project-sized chunks of the larger vision has already been both productive and personally rewarding.
I look forward to continuing work with them this year and for future mentorship/collaboration opportunities with younger students.
The relationship pays dividends both ways: more collaborators help advance the work, and younger students get exposed to research sooner (as I wish I had been).

\paragraph{Teaching}

As a student, I've always enjoyed classes that lift the curtain on systems that seem opaque.
So at CMU, I enjoyed TAing the notorious operating systems course twice.
Printing out and reading kernels was hard work, but many students had never undergone a code review before, so they found the feedback invaluable.

I've continued teaching in grad school. This year, I TAed the first course in the systems sequence at UW.
I also take part in the departmental undergrad tutoring program, where graduate students get paired with a small group of undergrads for additional academic help.

My research statement details how my proposed work will have broader impact in education.
Through my own experience as a student and teacher, I appreciate the value that efficient tooling adds to the learning experience.
I plan to open-source the resulting hardware and software so educators in chemistry and biology can do more experiments with less cost thanks to the automation at small scale.
I look forward to working with teachers and students to support such use cases.

\section*{Future Goals}

I plan to keep working to make programming easier, more efficient, and even possible complex domains.
But success in this includes more than just the research.
I want to take every opportunity to share and collaborate with domain experts, educators, fellow researchers, and younger students.
I want to present and support my work to ensure that it's understood and usable.
Only by taking these steps will I ensure that my research has the intended impact of empowering problem solvers in various domains.

With the help of an NSF Fellowship, I can follow this path through my PhD to my long term goal of professorship.
Leading a research lab will combine my research goals with my personal passion for mentoring.
An academic job will also allow me to continue working in the classroom and building tools for education.
An NSF Fellowship will let me not only pursue my research agenda, but further grow my research, education, and mentoring skills with the ultimate goal of passing it along to others.

% These goals offer many opportunities to collaborate with and mentor undergraduate students, which I am currently enjoying and plan to do even more.

% My long term goal of becoming a professor combines my research and personal experiences.

% This project requires collaboration with both hardware designers and synthetic biologists, and together we will prototype microfluidic chips and run real-world protocols on them.
% My proposed work on programmable fluidics and my collaboration with the MISL group at UW will place me squarely in the middle of this new field.

% My position in this exciting new area will attract a talented cohort of students, and I cannot wait to see what such a collaboration will bring.

% this project would also provide the automation necessary to advance the field of molecular computation and storage.
% Combined with advances in various technologies in synthetic biology, this will lead to an integrated platform which can store and manipulate data in both the silicon and molecular domains.

\vfill
\hrule
\renewcommand*{\bibfont}{\footnotesize}
\printbibliography[heading=none]

\end{document}
